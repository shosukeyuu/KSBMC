\documentclass[12pt,a4paper]{article}
\input{latexmacros.tex}

\title{Model Checker for the Logic of Knowledge and Safe Belief}
\author{Simon Chiu}
\date{\today}
\hypersetup{pdfauthor={Simon Chiu}, pdftitle={Model Checker for the Logic of Knowledge and Safe Belief}}

\begin{document}

\maketitle

\newpage

\tableofcontents

\clearpage

% We include one file for each section. The ones containing code should
% be called something.lhs and also mentioned in the .cabal file.

\section{Introduction}\label{sec:Introduction}
This project implements an explicit model checker for the logic of knowledge and safe belief for a single agent, $K\Box$, as presented in \cite{baltagQualitativeTheoryDynamic2008}, together with three dynamic belief revision operators: update or truthful public announcement, radical upgrade, and conservative upgrade, which we will call $K\Box +$. We will use the language Haskell, which provides a clean way to model and manipulate abstract structures.

\input{lib/Syntax.lhs}

\input{lib/Semantics.lhs}

\input{lib/Examples.lhs}

\input{test/simpletests.lhs}

\section{Future Work}\label{sec:Conclusion}
Currently, this model checker only implements three dynamic belief revision operators. Expanding this by implementing the full logic of doxastic actions, which includes the action-priority update, would allow this to be used for a wider range of models. Also, as this model checker can only be operated using a code-editor and through the command line, it would be nice to add a UI to make it more accessible to those who are not well-versed in Haskell. 

The full code can be found on \url{https://github.com/shosukeyuu/KSBMC}.

\addcontentsline{toc}{section}{Bibliography}
\bibliographystyle{alpha}
\bibliography{references.bib}

\end{document}
