\documentclass[12pt,a4paper]{article}
\usepackage{etex,datetime,setspace,latexsym,amssymb,amsmath,amsthm}
\usepackage{fancybox,dialogue,float,wrapfig,enumerate,microtype}
\usepackage{verbatim,xcolor,multicol,titlesec,tabularx,mdframed}

\usepackage[utf8]{inputenc}
\usepackage[pdftex]{hyperref}
\usepackage[margin=2cm,bottom=3cm,footskip=15mm]{geometry}
\parindent0cm
\parskip0.5em

\usepackage{tikz}
\usetikzlibrary{arrows,trees,positioning,shapes,patterns}
\usetikzlibrary{intersections,calc,fpu,decorations.pathreplacing}

\usepackage[T1]{fontenc} % better fonts

% Haskell code listings in our own style
\usepackage{listings,color}
\definecolor{lightgrey}{gray}{0.35}
\definecolor{darkgrey}{gray}{0.20}
\definecolor{lightestyellow}{rgb}{1,1,0.92}
\definecolor{dkgreen}{rgb}{0,.2,0}
\definecolor{dkblue}{rgb}{0,0,.2}
\definecolor{dkyellow}{cmyk}{0,0,.7,.5}
\definecolor{lightgrey}{gray}{0.4}
\definecolor{gray}{gray}{0.50}
\lstset{
  language        = Haskell,
  basicstyle      = \scriptsize\ttfamily,
  keywordstyle    = \color{dkblue},     stringstyle     = \color{red},
  identifierstyle = \color{dkgreen},    commentstyle    = \color{gray},
  showspaces      = false,              showstringspaces= false,
  rulecolor       = \color{gray},       showtabs        = false,
  tabsize         = 8,                  breaklines      = true,
  xleftmargin     = 8pt,                xrightmargin    = 8pt,
  frame           = single,             stepnumber      = 1,
  aboveskip       = 2pt plus 1pt,
  belowskip       = 8pt plus 3pt
}
\lstnewenvironment{code}[0]{}{}

% only shown, not compiled:
\lstnewenvironment{showCode}[0]{\lstset{numbers=none}}{}

% only compiled, not shown:
\newcommand{\hide}[1]{}

% will the real phi please stand up
\renewcommand{\phi}{\varphi}

%added macros
\newcommand{\Prop}{\mathsf{Prop}}
\newcommand{\M}{\mathcal{M}}
\newcommand{\Min}{\text{Min}}

% load hyperref as late as possible for compatibility
\usepackage[pdftex]{hyperref}
\hypersetup{
  pdfborder = {0 0 0},
  breaklinks = true,
  linktoc = all,
}
\pdfinfoomitdate=1
\pdftrailerid{}
\pdfsuppressptexinfo15


\title{Model Checker for the Logic of Knowledge and Safe Belief}
\author{Simon Chiu}
\date{\today}
\hypersetup{pdfauthor={Simon Chiu}, pdftitle={Model Checker for the Logic of Knowledge and Safe Belief}}

\begin{document}

\maketitle

\newpage

\tableofcontents

\clearpage

% We include one file for each section. The ones containing code should
% be called something.lhs and also mentioned in the .cabal file.

\section{Introduction}\label{sec:Introduction}
This project implements an explicit model checker for the logic of knowledge and safe belief for a single agent, $K\Box$, as presented in \cite{baltagQualitativeTheoryDynamic2008}, together with three dynamic belief revision operators: update or truthful public announcement, radical upgrade, and conservative upgrade, which we will call $K\Box +$. We will use the language Haskell, which provides a clean way to model and manipulate abstract structures.

\input{lib/Syntax.lhs}

\input{lib/Semantics.lhs}

\input{lib/Examples.lhs}

\input{test/simpletests.lhs}

\section{Future Work}\label{sec:Conclusion}
Currently, this model checker only implements three dynamic belief revision operators. Expanding this by implementing the full logic of doxastic actions, which includes the action-priority update, would allow this to be used for a wider range of models. Also, as this model checker can only be operated using a code-editor and through the command line, it would be nice to add a UI to make it more accessible to those who are not well-versed in Haskell. 

The full code can be found on \url{https://github.com/shosukeyuu/KSBMC}.

\addcontentsline{toc}{section}{Bibliography}
\bibliographystyle{alpha}
\bibliography{references.bib}

\end{document}
